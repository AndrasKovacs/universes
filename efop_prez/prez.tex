%% build: latexmk -pdf -pvc -xelatex prez.tex

\documentclass[dvipsnames,10pt,aspectratio=169]{beamer}
\usetheme{metropolis}           % Use metropolis theme

\usepackage{xcolor}
\usepackage{amssymb}
\usepackage{amsmath}
\usepackage{amsfonts}
\usepackage{fontspec}
\usepackage{proof}
\usepackage{tikz-cd}
\usepackage{mathpartir}
\usepackage{stmaryrd}
\usepackage{hyperref}
\hypersetup{colorlinks,linkcolor=blue,urlcolor=blue}
\setmonofont[Scale=0.7]{DejaVu Sans Mono}


\bibliographystyle{alpha}
\setbeamerfont{bibliography item}{size=\footnotesize}
\setbeamerfont{bibliography entry author}{size=\footnotesize}
\setbeamerfont{bibliography entry title}{size=\footnotesize}
\setbeamerfont{bibliography entry location}{size=\footnotesize}
\setbeamerfont{bibliography entry note}{size=\footnotesize}
\setbeamertemplate{bibliography item}{}


%% --------------------------------------------------------------------------------

\title{Univerzumok Típuselméletben}
\date{2021 április 8., EFOP Workshop}
\author{\normalsize{\vspace{-1em}\textbf{Kovács András}\footnote{A szerzőt az Európai Unió támogatta, közös finanszírozásban az Európai Szociális Alappal (EFOP-3.6.3-VEKOP-16-2017-00002).\vspace{0.5em}}}}
%% \institute{Eötvös Loránd University, Budapest}
\begin{document}
\maketitle

%% ------------------------------------------------------------

\begin{frame}{Russell paradoxon és méret-problémák}

Nincs olyan $S$ halmaz egy konzisztens halmazelméletben, hogy $x \in S$ pontosan akkor, ha $x \notin
x$.
\vspace{1em}

Alternatív fogalmazás:
\begin{itemize}
  \item Borbély paradoxon
  \item Összes halmaz halmaza
\end{itemize}
\vspace{1em}

Kapcsolódó:
\begin{itemize}
  \item Valós számok nem megszámlálhatóak
  \item Gödel első nem-teljességi tétele
  \item Megállási probléma
\end{itemize}
\vspace{1em}

{\footnotesize (Általánosításuk: \href{https://arxiv.org/pdf/math/0305282.pdf}{Lawvere fixpont tétele)}}
\end{frame}

%% ------------------------------------------------------------

\begin{frame}{Megoldás univerzumokkal}

Csak \textbf{létező halmaz részhalmazát} definiálhatjuk logikai formulával.
\vspace{1em}

De olykor szeretnénk kvantifikálni egy halmazelmélet \textbf{minden halmaza}
fölött.
\vspace{1em}

\textbf{Univerzum:} halmaz, ami tartalmazza egy halmazelmélet (pl. ZF) összes halmazát.
\vspace{1em}

Gödel, Russell: ZF-ben nincs univerzum, ami az összes ZF halmazt tartalmazza.
\vspace{1em}

Új univerzumokat axiómaként kell felvenni.
\end{frame}


\begin{frame}[fragile]{Típuselméletben}

Típuselméletek:
\begin{itemize}
  \item A halmazelméletek alternatívái.
  \item Tételbizonyító-rendszerek gyakori alapja.
  \item Bizonyításra és programozásra egyszerre alkalmas.
\end{itemize}

Formális vs. informális univerzum kezelés.
\vspace{1em}

Létező bizonyító-rendszerekben: univerzum megvalósítások sokasága kisebb-nagyobb eltérésekkel.
\vspace{1em}

Szeretnénk ezeket lefedni egy általános szemantikával.

\end{frame}

\begin{frame}[fragile]{Példa}

Agda:
\begin{verbatim}
  ℕ             : Set₀
  Set₀          : Set₁
  Set₁          : Set₂
  (ℕ → Set₀)    : Set₁
  (Set₀ → Bool) : Set₁
\end{verbatim}

Identitás függvény minden ``kis'' halmazon:
\begin{verbatim}
  id : (A : Set₀) → A → A
  id A x = x
\end{verbatim}

Minden $i$-re $\mathsf{Set_i} : \mathsf{Set_{i + 1}}$.
\vspace{1em}

Ellentmondás lenne: $\mathsf{Set_i} : \mathsf{Set_{i}}$.
\vspace{1em}

\end{frame}

\begin{frame}[fragile]{Kiterjesztések}

Az alap rendszer vesződséges tud lenni:
\begin{verbatim}
  id₀ : (A : Set₀) → A → A
  id₁ : (A : Set₁) → A → A
  ...
\end{verbatim}


Coq/Agda/Lean-ben extra feature-ök vannak.
Pl. univerzum polimorfizmus Agda-ban:
\begin{verbatim}
  id : (l : Level)(A : Set l) → A → A
  id l A x = x
\end{verbatim}


\end{frame}

\begin{frame}[fragile]{Variációk}

\textbf{Mennyi univerzum?} Agda/Coq: megszámlálható.
\vspace{1em}

\textbf{Univerzumok teljesen rendzettek?} Agda/Coq: igen.
\vspace{1em}

\textbf{Milyen univerzum polimorfizmus?}. Coq: korlátos polimorfizmus. Agda: nincsenek
korlátok. Példa korláttal:
\begin{verbatim}
  myId : (i : ℕ) → i < 3 → (A : Set i) → A → A
  myId i p A x = x
\end{verbatim}

\textbf{Milyen műveletek vannak univerzum-szinteken?} Az Agda liberálisabb a Coq-nál:
\begin{verbatim}
  ℕtoLevel : ℕ → Level
\end{verbatim}


\textbf{Kumulativitás.} Agda: nincs. Coq: van. Kumulativitás: ha \texttt{A : Set i}, akkor \texttt{A
  : Set (i + 1)}.

\end{frame}

\begin{frame}{Kutatási célok}

Szeretnénk belátni, hogy a különféle variációk mind jól viselkednek.
\vspace{1em}

Jól viselkedés:
\begin{itemize}
  \item Logikai konzisztencia.
  \item Igaz, hogy a típuselmélet egy programozási nyelv? Azaz: programok le kell, hogy
        fussanak elakadás nélkül. (Nincs jelenlegi munkában)
  \item Eldönthető a bizonyítások validálása? (Nincs jelenlegi munkában)
\end{itemize}

Egy általános rendszert használunk, ahol sok feature modellezhető, és
egyszerre bizonyítjuk az összes konzisztenciáját.

\end{frame}


\begin{frame}[fragile]{Amit a rendszerünk lefed}

Univerzum-szintek \textbf{bármilyen jól-rendezett halmazt} alkothatnak.
\begin{verbatim}
  Set i : Set (i + 1) : ... : Set ω : Set (ω + 1) : ...
\end{verbatim}
\textbf{Szintek fölötti kvantifikáció}, \textbf{tetszőleges számítások szinteken}.
\begin{verbatim}
   myId : (i : ℕ) → (j : ℕ) → (A : Set (i + j)) → A → A
   myId i j A x = x
\end{verbatim}
\textbf{Kumulativitás}
\begin{verbatim}
   ℕ : Set₀
   ℕ : Set₁
   ...
\end{verbatim}

\end{frame}

\begin{frame}[fragile]{Megvalósítás}

Visszavezetjük az összes feature-t egy már ismert feature-re,
az \textbf{induktív-rekurzív} definíciókra.
\vspace{1em}

Az indukció-rekurzió egy típuselméleti változata a halmazelméleti \textbf{Mahlo kardinálisoknak}.

\begin{verbatim}
   Univ   : Set
   Nat    : Univ
   Π      : (A : Univ) → (Interp Univ → Univ) → Univ

   Interp : Univ → Set
   Interp Nat     = ℕ
   Interp (Π A B) = (x : Interp A) → Interp (B x)
\end{verbatim}

\end{frame}


\begin{frame}{Publikáció}

``Generalized Universe Hierarchies and First-Class Universe Levels''
\vspace{1em}

Elbírálás alatt FSCD 2021 konferenciára.

%% This talk is a side project intended for FSCD 2021 submission (in February).
%% \vspace{1em}

%% Other 2020 publications:
%% \begin{itemize}
%%   \item LICS 2020, with Ambrus: ``Large and Infinitary Quotient Inductive-Inductive Types''
%%   \item ICFP 2020: ``Elaboration with First-Class Implicit Function Types''
%%   \item TYPES 2019 post-proceedings, with Ambrus and Ambroise Lafont: ``For
%%     Finitary Induction-Induction, Induction is Enough''
%% \end{itemize}

\end{frame}

\begin{frame}

\begin{center} {\large Köszönöm a figyelmet!} \end{center}

\end{frame}

%% \begin{frame}[allowframebreaks]{References}
%%   \bibliography{references}
%% \end{frame}

\end{document}
