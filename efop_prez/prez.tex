%% build: latexmk -pdf -pvc -xelatex prez.tex

\documentclass[dvipsnames,10pt,aspectratio=169]{beamer}
\usetheme{metropolis}           % Use metropolis theme

\usepackage{xcolor}
\usepackage{amssymb}
\usepackage{amsmath}
\usepackage{amsfonts}
\usepackage{fontspec}
\usepackage{proof}
\usepackage{tikz-cd}
\usepackage{mathpartir}
\usepackage{stmaryrd}
\usepackage{hyperref}
\hypersetup{colorlinks,linkcolor=blue,urlcolor=blue}
\setmonofont[Scale=0.7]{DejaVu Sans Mono}


\bibliographystyle{alpha}
\setbeamerfont{bibliography item}{size=\footnotesize}
\setbeamerfont{bibliography entry author}{size=\footnotesize}
\setbeamerfont{bibliography entry title}{size=\footnotesize}
\setbeamerfont{bibliography entry location}{size=\footnotesize}
\setbeamerfont{bibliography entry note}{size=\footnotesize}
\setbeamertemplate{bibliography item}{}


%% --------------------------------------------------------------------------------

\title{Universes In Type Theory}
\date{April 8, 2021, EFOP Workshop}
\author{\normalsize{\vspace{-1em}\textbf{András Kovács}\footnote{The author was supported by the European Union, co-financed by the
European Social Fund (EFOP-3.6.3-VEKOP-16-2017-00002).\vspace{0.5em}}}}
%% \institute{Eötvös Loránd University, Budapest}
\begin{document}
\maketitle

%% ------------------------------------------------------------

\begin{frame}{Russell's paradox, or size issues}

There is no set $S$ in a consistent set theory such that $x \in S$ iff $x \notin x$.
\vspace{1em}

Alternate formulations:
\begin{itemize}
  \item Barber's paradox
  \item No set of all sets
\end{itemize}
\vspace{1em}

Related:
\begin{itemize}
  \item Real numbers are uncountable
  \item Gödel's first incompleteness theorem
  \item Undecidability of halting
\end{itemize}
\vspace{1em}

{\footnotesize (All instances of \href{https://arxiv.org/pdf/math/0305282.pdf}{Lawvere's fixed point theorem)}}
\end{frame}

%% ------------------------------------------------------------

\begin{frame}{Solution with universes}

Solving Russell's paradox: set comprehension can only define a \textbf{subset} of a set.
\vspace{1em}

But sometimes we still want to quantify over \textbf{all sets} of
Zermelo-Fraenkel set theory.
\vspace{1em}

\textbf{Universe:} a set which includes every ZF set.
\vspace{1em}

Gödel: there is no universe inside ZF.
\vspace{1em}

We can assume more universes if we want to quantify over more sets.
\end{frame}


\begin{frame}[fragile]{Universes in type theory}

Type theories:
\begin{itemize}
  \item Alternative to set theory as mathematical foundation.
  \item Used in most general-purpose theorem provers.
  \item Can be used for proving and programming at the same time.
\end{itemize}

Universes come up in type theories as well.
\vspace{1em}

Every theorem prover has different universe-related features.
\vspace{1em}

We want to give a general semantics covering many different features.

\end{frame}

\begin{frame}[fragile]{Example}

Agda:
\begin{verbatim}
  ℕ             : Set₀
  Set₀          : Set₁
  Set₁          : Set₂
  (ℕ → Set₀)    : Set₁
  (Set₀ → Bool) : Set₁
\end{verbatim}

Identity function at all small sets:
\begin{verbatim}
  id : (A : Set₀) → A → A
  id A x = x
\end{verbatim}

We have $\mathsf{Set_i} : \mathsf{Set_{i + 1}}$ for all $i$.
\vspace{1em}

Assuming $\mathsf{Set_i} : \mathsf{Set_{i + 1}}$ implies contradiction (by Russell-like argument).
\vspace{1em}

\end{frame}

\begin{frame}[fragile]{Extensions}

The basic system can be tedious:
\begin{verbatim}
  id₀ : (A : Set₀) → A → A
  id₁ : (A : Set₁) → A → A
  ...
\end{verbatim}

In Coq/Agda/Lean, various extra features are used.
Example: universe polymorphism in Agda:
\begin{verbatim}
  id : (l : Level)(A : Set l) → A → A
  id l A x = x
\end{verbatim}


\end{frame}

\begin{frame}[fragile]{Design choices and variations}

\textbf{How many universes?} Agda/Coq: countably many.
\vspace{1em}

\textbf{Are universes totally ordered?} Agda/Coq: yes.
\vspace{1em}

\textbf{What kind of level polymorphism?}. Coq: bounded polymorphism. Agda: no
bounds allowed. Bounded example:
\begin{verbatim}
  myId : (i : ℕ) → i < 3 → (A : Set i) → A → A
  myId i p A x = x
\end{verbatim}

\textbf{What kind of operations are available on levels?} Agda is more liberal than Coq. Example:
\begin{verbatim}
  ℕtoLevel : ℕ → Level
\end{verbatim}


\textbf{Are universes cumulative} Agda: no. Coq: yes. Cumulativity: whenever \texttt{A : Set i}, we also have \texttt{A : Set (i + 1)}.

\end{frame}

\begin{frame}{Research goals}

We want to know that each point in the design space makes sense.
\vspace{1em}

Making sense:
\begin{itemize}
  \item Logical consistency.
  \item Is the type theory a proper programming language? Programs should compute to values
        and not get randomly stuck.
  \item Is proof checking decidable? (Not covered in current work).
\end{itemize}

Approach: use generic framework to cover as many features/variations as
possible. Prove that everything in the framework is sensible.


\end{frame}


\begin{frame}[fragile]{Features covered by the framework}

Universe levels may come from \textbf{any well-ordered set}, even transfinite:
\begin{verbatim}
  Set i : Set (i + 1) : ... : Set ω : Set (ω + 1) : ...
\end{verbatim}
\textbf{Quantification over levels}, \textbf{arbitrary computation on levels}.
\begin{verbatim}
   myId : (i : ℕ) → (j : ℕ) → (A : Set (i + j)) → A → A
   myId i j A x = x
\end{verbatim}
\textbf{Universes are cumulative}
\begin{verbatim}
   ℕ : Set₀
   ℕ : Set₁
   ...
\end{verbatim}

\end{frame}

\begin{frame}[fragile]{Implementation}

We define a family of type theories with general universe features.
\vspace{1em}

We prove all of them consistent by reducing all fancy features to
a single previously known feature, called \textbf{induction-recursion}.
\vspace{1em}

Induction-recursion is a type-theoretic analogue of assuming \textbf{Mahlo cardinals} in set theory.

\begin{verbatim}
   Univ   : Set
   Nat    : Univ
   Π      : (A : Univ) → (Interp Univ → Univ) → Univ

   Interp : Univ → Set
   Interp Nat     = ℕ
   Interp (Π A B) = (x : Interp A) → Interp (B x)
\end{verbatim}


\end{frame}





%% \begin{enumerate}
%%   \item Universe levels may come from any well-ordered set.
%%   \item Polymorphism with bounds is allowed.
%%   \item Universes are cumulative.
%%   \item Levels can be manipulated arbitrarily as program data (WIP).
%% \end{enumerate}

%% Things we need to assume in order to prove the framework sensible:
%% \begin{enumerate}
%%   \item \textbf{Two} universes.
%%   \item Types of certain infinitely branching trees (W-types).
%% \end{enumerate}

%% Potential further applications (in future work):
%% \begin{itemize}
%%   \item Information flow type systems (``secure'' and ``public'' levels).
%%   \item Staged compilation (``runtime'' and ``compile time'' levels).
%% \end{itemize}

%% \end{frame}

\begin{frame}{Publication}

``Generalized Universe Hierarchies and First-Class Universe Levels''
\vspace{1em}

Submission under review for FSCD 2021.


%% This talk is a side project intended for FSCD 2021 submission (in February).
%% \vspace{1em}

%% Other 2020 publications:
%% \begin{itemize}
%%   \item LICS 2020, with Ambrus: ``Large and Infinitary Quotient Inductive-Inductive Types''
%%   \item ICFP 2020: ``Elaboration with First-Class Implicit Function Types''
%%   \item TYPES 2019 post-proceedings, with Ambrus and Ambroise Lafont: ``For
%%     Finitary Induction-Induction, Induction is Enough''
%% \end{itemize}

\end{frame}


\begin{frame}

\begin{center} {\large Thank you!} \end{center}

\end{frame}


%% \begin{frame}[allowframebreaks]{References}
%%   \bibliography{references}
%% \end{frame}

\end{document}
