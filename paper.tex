
\documentclass[a4paper,UKenglish,cleveref, autoref, thm-restate]{lipics-v2021}
%This is a template for producing LIPIcs articles.
%See lipics-v2021-authors-guidelines.pdf for further information.
%for A4 paper format use option "a4paper", for US-letter use option "letterpaper"
%for british hyphenation rules use option "UKenglish", for american hyphenation rules use option "USenglish"
%for section-numbered lemmas etc., use "numberwithinsect"
%for enabling cleveref support, use "cleveref"
%for enabling autoref support, use "autoref"
%for anonymousing the authors (e.g. for double-blind review), add "anonymous"
%for enabling thm-restate support, use "thm-restate"
%for enabling a two-column layout for the author/affilation part (only applicable for > 6 authors), use "authorcolumns"
%for producing a PDF according the PDF/A standard, add "pdfa"

%\graphicspath{{./graphics/}}%helpful if your graphic files are in another directory

\bibliographystyle{plainurl}% the mandatory bibstyle

\title{Generalized Universe Hierarchies and First-Class Universe Levels}
%% \titlerunning{Dummy short title} %TODO optional, please use if title is longer than one line

\author{András Kovács}{Eötvös Loránd University, Hungary}{kovacsandras@inf.elte.hu}{https://orcid.org/0000-0002-6375-9781}{}

%TODO mandatory, please use full name; only 1 author per \author macro; first two parameters are mandatory, other parameters can be empty. Please provide at least the name of the affiliation and the country. The full address is optional
\authorrunning{A., Kovács} %TODO mandatory. First: Use abbreviated first/middle names. Second (only in severe cases):

\Copyright{András Kovács} %TODO mandatory, please use full first names. LIPIcs license is "CC-BY";  http://creativecommons.org/licenses/by/3.0/

\begin{CCSXML}
<ccs2012>
<concept>
<concept_id>10003752.10003790.10011740</concept_id>
<concept_desc>Theory of computation~Type theory</concept_desc>
<concept_significance>500</concept_significance>
</concept>
</ccs2012>
\end{CCSXML}

\ccsdesc[500]{Theory of computation~Type theory}

\keywords{type theory, universes} %TODO mandatory; please add comma-separated list of keywords

%% \category{} %optional, e.g. invited paper

\relatedversion{} %optional, e.g. full version hosted on arXiv, HAL, or other respository/website
%\relatedversiondetails[linktext={opt. text shown instead of the URL}, cite=DBLP:books/mk/GrayR93]{Classification (e.g. Full Version, Extended Version, Previous Version}{URL to related version} %linktext and cite are optional

%\supplement{}%optional, e.g. related research data, source code, ... hosted on a repository like zenodo, figshare, GitHub, ...
%\supplementdetails[linktext={opt. text shown instead of the URL}, cite=DBLP:books/mk/GrayR93, subcategory={Description, Subcategory}, swhid={Software Heritage Identifier}]{General Classification (e.g. Software, Dataset, Model, ...)}{URL to related version} %linktext, cite, and subcategory are optional

\funding{The author was supported by the European Union,
co-financed by the European Social Fund (EFOP-3.6.3-VEKOP-16-2017-00002).}%optional, to capture a funding statement, which applies to all authors. Please enter author specific funding statements as fifth argument of the \author macro.


%\nolinenumbers %uncomment to disable line numbering

%\hideLIPIcs  %uncomment to remove references to LIPIcs series (logo, DOI, ...), e.g. when preparing a pre-final version to be uploaded to arXiv or another public repository

%Editor-only macros:: begin (do not touch as author)%%%%%%%%%%%%%%%%%%%%%%%%%%%%%%%%%%
\EventEditors{John Q. Open and Joan R. Access}
\EventNoEds{2}
\EventLongTitle{42nd Conference on Very Important Topics (CVIT 2016)}
\EventShortTitle{CVIT 2016}
\EventAcronym{CVIT}
\EventYear{2016}
\EventDate{December 24--27, 2016}
\EventLocation{Little Whinging, United Kingdom}
\EventLogo{}
\SeriesVolume{42}
\ArticleNo{23}


%% --------------------------------------------------------------------------------


\usepackage{xcolor}
\usepackage{mathpartir}
\usepackage{todonotes}
\presetkeys{todonotes}{inline}{}
\usepackage{scalerel}
\usepackage{amssymb}


%% \theoremstyle{definition}
%% \newtheorem{definition}{Definition}
%% \newtheorem{example}{Example}

%% \theoremstyle{theorem}
%% \newtheorem{theorem}{Theorem}
%% \newtheorem{lemma}{Lemma}

\theoremstyle{remark}
\newtheorem{notation}{Notation}

\theoremstyle{definition}
\newtheorem{mydefinition}{Definition}
\newtheorem{myexample}{Example}


%% Abbrevs
%% --------------------------------------------------------------------------------

\newcommand{\Set}[1]{\mathsf{Set_{#1}}}
\newcommand{\Seti}{\mathsf{Set}}
\newcommand{\refl}{\mathsf{refl}}
\newcommand{\Con}{\mathsf{Con}}
\newcommand{\Ty}{\mathsf{Ty}}
\newcommand{\Tm}{\mathsf{Tm}}
\newcommand{\Sub}{\mathsf{Sub}}
\newcommand{\emptycon}{\scaleobj{.75}\bullet}
\newcommand{\U}{\mathsf{U}}
\newcommand{\El}{\mathsf{El}}
\newcommand{\id}{\mathsf{id}}
\newcommand{\ext}{\triangleright}
\newcommand{\blank}{\mathord{\hspace{1pt}\text{--}\hspace{1pt}}}
\newcommand{\mi}[1]{\mathit{#1}}
\newcommand{\p}{\mathsf{p}}
\newcommand{\q}{\mathsf{q}}

\renewcommand{\tt}{\mathsf{tt}}


%% --------------------------------------------------------------------------------
\begin{document}
\maketitle

\begin{abstract}
In type theories, universe hierarchies are commonly used to increase the
expressive power of the theory while avoiding inconsistencies arising from size
issues. There are numerous ways to specify universe hierarchies, and theories
may differ in details of cumulativity, choice of universe levels, specification
of type formers and eliminators, and available internal operations on levels. In
the current work, we aim to provide a framework which covers a large part of the
design space. First, we develop syntax and semantics for cumulative universe
hierarchies, where levels may come from any set equipped with a transitive
well-founded ordering. In the semantics, we show that induction-recursion can be
used to model transfinite hierarchies, and also support lifting operations on
type codes which strictly preserve type formers. Then, we consider a setup where
universe levels are first-class types and subject to arbitrary internal
reasoning. This generalizes the bounded universe polymorphism features of Coq
and also the internal level computations in Agda.
\end{abstract}

\section{Introduction}
\label{sec:introduction}

Users of type theories often view universe levels as an unwieldy bureaucratic
detail, a necessary annoyance in service of boosting expressive power while
retaining logical consistency. However, universe hierarchies are not going away
any time soon in practical implementations of type theory. In recent
developments of systems, we are getting more universes and more adjacent
features:

\begin{itemize}
\item Agda recently added a limited version cumulativity as an optional feature
  for universes, and the upcoming 2.6.2 version will extend the $\omega+1$
  universe hierarchy to $\omega*2$.
\item Coq added support for cumulative inductive types \cite{timany18cumulative}, and added a form of
  bounded universe polymorphism \cite{ziliani15unification}.
\end{itemize}

\noindent At this point, there is a veritable zoo of universe features in existing
implementations. We have perhaps even more design choices when considering the
formal metatheory of type theories. Do type formers stay in the same
universe, or take the $\sqcup$ of universes of constituent types? Can
eliminators target any universe, or do we instead use lifting operators to cross
levels? What kind of universe polymorphism do we have, can we quantify over
bounds? Is there a type of levels, or are levels in a separate syntactic layer?

The aim of the current work is to develop semantics which covers as much as
possible from the range of sensible universe features. This way, theorists and
language implementors can grab a desired bag of features, and be able to show
consistency of their system by a straightforward translation to one of the
systems in this paper.

\paragraph*{Contributions}

\begin{enumerate}
\item In Section \ref{sec:ttgu} we describe models of type theories where
  universe levels may come from any set with a well-founded transitive ordering
  relation. We specify models as categories equipped with level-indexed diagrams
  of families, as a variation on categories with families. Each morphism of
  levels is mapped to a lifting operation on terms and types. By varying the
  preservation properties of lifting operations, we can describe a range of
  stratification features, from two-level type theory to cumulative universes.
\item We use induction-recursion to model the mentioned theories. Here, we model
  the strongest formulations for lifting and universes, namely cumulative
  universes with Russell-style type decoding. The main challenge is combining
  transfinite hierarchies with strictly type-preserving lifting operations.
\item In Section \ref{sec:ttfl} we describe type theories with internal types
  for levels and level morphisms, and extend the previous inductive-recursive
  semantics to cover these as well. Here, we can additionally represent various
  universe polymorphism features and level computations. We illustrate how
  predicative universe features in Coq and Agda could be justified in our framework.
\end{enumerate}

\noindent Several related topics are not handled in this paper.

\begin{itemize}

\item
Besides consistency, we are often interested in \emph{canonicity},
\emph{normalization} or other metatheoretical properties. The current work
focuses on consistency and leaves other properties to future work. However, we
do keep canonicity in mind while specifying our systems, and aim to avoid
pitfalls which would cause canonicity to fail.
\item
We only focus on using universes as sized-based classifiers for
types. Universe-like stratification is also present in two-level type theory
\cite{twolevel}, modal type theories \cite{gratzer20multimodal} or as h-levels
in homotopy type theory \cite{hottbook}, but we do not attempt to cover such use
cases.
\item
We only cover predicative hierarchies.
\end{itemize}

We provide an Agda formalization of the contents of the paper. The formalization
is not complete, as we skip the proofs involving an over-abundance of equality
coercions (which are more suited to informal reasoning, using equality
reflection), and instead focus on the key points.

\section{Metatheory}
\label{sec:metatheory}

We work in a Martin-Löf type which has the following features.
\begin{itemize}
  \item Two universes named $\Set0$ and $\Set1$, where $\Set0$ supports
    inductive-recursive types (IR) as specified by Dybjer and Setzer
    \cite{dybjer99finite}. The $\Set1$ universe is not essential and we only use
    it as a convenience feature, in this paper and in the Agda formalization. We
    may omit the universe indices if they can be inferred or if we work over
    arbitrary indices.
  \item Function extensionality and uniqueness of identity proofs
    (UIP). Additionally, we assume equality reflection in this paper, thus
    working in extensional type theory, to avoid noise from equality transports.
  \item We write function types as $(x : A)\to B$ with $\lambda\,x.\,t$
    inhabitants, and $\Sigma$-types as $(x : A) \times B$, with pairing as
    $(t,\, u)$ and projections as $t.1$ and $t.2$. We have $\top$ as the unit
    type with inhabitant $\tt$, and $\bot$ as the empty type. Propositional identity
    is written as $t = u$ with $\refl : t = t$.
  \item We occasionally use $\{x : A\} \to B$ for an Agda-like notation for
    function types with implicit arguments, and omit implicit applications
    whenever they can be inferred.

\end{itemize}

\section{Type Theories with Generalized Universe Hierarchies}
\label{sec:ttgu}

In this section, we first describe notions of models for type theories with
generalized universes, and discuss several variations of universes and lifting
operations. Then, we pick a concrete variant (the ``strongest'', in a sense)
and construct a model for it in the metatheory.

For the basic structure of typing contexts and substitutions, let us review
categories with families.

\begin{mydefinition}[Category with family (CwF), c.f. \cite{Dybjer96internaltype}]\label{def:cwf}
A CwF consists of the following data:
\begin{itemize}
\item A category with a terminal object. We denote the set of objects as $\Con :
  \Seti$ and use capital Greek letters starting from $\Gamma$ to refer to
  objects. The set of morphisms is $\Sub : \Con \to \Con \to \Seti$, and we use
  $\sigma$, $\delta$ and so on to refer to morphisms. The terminal object is
  $\emptycon$ with unique morphism $\epsilon : \Sub\,\Gamma\,\emptycon$. In
  initial models (that is, syntaxes) of type theories, objects correspond to
  typing contexts, morphisms to parallel substitutions and the terminal object to
  the empty context; this informs the naming scheme that we use here.
\item A \emph{family structure}, containing $\Ty : \Con \to \Seti$ and $\Tm :
  (\Gamma : \Con) \to \Ty\,\Gamma \to \Seti$, where $\Ty$ is a presheaf over the
  category of contexts and $\Tm$ is a presheaf over the category of elements of
  $\Ty$. This means that both types ($\Ty$) and terms ($\Tm$) can be
  substituted, and substitution has functorial action. We use $A$, $B$, $C$ to
  refer to types and $t$, $u$, $v$ to refer to terms, and use $A[\sigma]$ and
  $t[\sigma]$ for substituting types and terms. Additionally, a family
  structure has \emph{context comprehension} which consists of an operation
  $\blank\ext\blank : (\Gamma : \Con) \to \Ty\,\Gamma \to \Con$ together with an
  isomorphism $\Sub\,\Gamma\,(\Delta\ext A) \simeq ((\sigma : \Sub\,\Gamma\,\Delta)
  \times \Tm\,\Gamma\,(A[\sigma]))$ which is natural in $\Gamma$.
\end{itemize}
\end{mydefinition}

\noindent From the comprehension structure, we recover the following notions:

\begin{itemize}
\item By going right-to-left along the isomorphism, we recover \emph{substitution extension}
      $\blank,\blank : (\sigma : \Sub\,\Gamma\,\Delta) \to \Tm\,\Gamma\,(A[\sigma])$. This means
      that starting from $\epsilon$ or the identity substitution $\id$, we can iterate $\blank,\blank$
      to build substitutions as lists of terms.
\item By going left-to-right, and starting from $\id : \Sub\,(\Gamma\ext A)\,(\Gamma\ext A)$, we recover
      the \emph{weakening substitution} $\p : \Sub\,(\Gamma\ext A)\,\Gamma$ and the \emph{zero variable}
      $\q : \Tm\,(\Gamma\ext A)\,(A[\p])$.
\item By weakening $\q$, we recover a notion of variables as De Bruijn indices. In general, the $n$-th
      De Bruijn index is defined as $\q[\p^{n}]$, where $\p^{n}$ denotes $n$-fold composition.
\end{itemize}

notions of substitution extension, weakening
and variables, internally to any CwF.



\subsection{Categories with Families}
\label{sec:categories_with_families}

\subsection{Morphisms, Lifts and Inclusions of Families}
\label{sec:morphisms}

morphisms, lifts (2ltt), inclusions, strict inclusion (cumulative)

\subsection{Level Structures}
\label{sec:level_structures}

Level structures, thin direct semicat, vs thin cat, direct semicat, direct cat

\subsection{Universes}
\label{sec:universes}

Coquand, Russell, naturality, large elimination examples
Lifting of universes
No lifting naturality for (El,Code)!

\section{Semantics}
\label{sec:semantics}

\subsection{Inductive-Recursive Codes}
\label{sec:inductive_recursive_codes}

IR codes, code lifting, properties,

\subsection{Model}

List a bunch of things in the model.

\section{Type Theories with First-Class Universe Levels}
\label{sec:ttfl}

Specification of the theory



%% --------------------------------------------------------------------------------


\bibliography{references}
\end{document}
